\documentclass[a4,8pt]{article}
\usepackage[margin=3cm,includefoot]{geometry}
\usepackage{siunitx}
\usepackage{fancyhdr}
\usepackage{tikz}
	\usetikzlibrary{arrows,shapes,positioning,calc}
\usepackage[font=it]{caption}
\usepackage{amsmath}
\usepackage{float}
\usepackage{pst-electricfield}

\pagestyle{fancy}
\lfoot{\sc Charlie Haddon}

\title{G485: Fields, Particles and Frontiers of Physics}
\author{Charlie Haddon}

\begin{document}
\begin{titlepage}
	\pagenumbering{}
	\maketitle
	\vspace{15cm}
\end{titlepage}
\clearpage

\pagenumbering{arabic}

\section{Electric and Magnetic Fields}
\subsection{Electric Fields}
Objects which have electric charge create a field around themselves called an
electric field which can exert forces on other charged particles. The strength 
of an electric field is measured force per unit positive charge. Electric fields
can be represented graphically by drawing lines showing the direction of the
field at different points. Arrows always point from positive to negative. 
(\SI{}{\newton\per\coulomb}).

\begin{figure}[H]
\begin{center}
\begin{pspicture*} (-6,-4) (6,4)
    \psElectricfield[Q={[1 -2 0][-1 2 0]},linecolor=black]
\end{pspicture*}
\end{center}
\caption{The electric field between a positive charge (left) and a negative
         charge (right)}
\end{figure}

The force on a charged particle due to an electric field is given by Coulomb's
law: 
$$F=\frac{Qq}{4\pi \epsilon_0 r^2}$$i
and therefore the electric field strength (force per unit charge) is given by:
$$E=\frac{Q}{4\pi \epsilon_0 r^2}$$

Between two parallel plates the electric field is uniform and given by 
$E=\frac{V}{d}$

\begin{figure}[H]
\begin{center}
\begin{pspicture*} (0,0) (0,0)
\end{pspicture*}
\end{center}
\caption{The electric field between a positive charge (left) and a negative
         charge (right)}
\end{figure}


\end{document}
