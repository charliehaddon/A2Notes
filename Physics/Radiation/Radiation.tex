\documentclass[a4,8pt]{article}
\usepackage[margin=3cm,includefoot]{geometry}
\usepackage{siunitx}
\usepackage{fancyhdr}
\usepackage{tikz}
	\usetikzlibrary{arrows,shapes,positioning,calc}
\usepackage[font=it]{caption}
\usepackage{amsmath}
\usepackage{float}

%\pagestyle{fancy}
\lfoot{\sc Charlie Haddon}

\title{Radiation}
\author{Charlie Haddon}

\begin{document}

\section{Uses of Radioisotopes}
Radioisotopes have several common uses in industry and in homes including the
following:

\begin{itemize}
    \item Smoke detectors use radioisotopes to ionise air (typically 
    americium-241). Two chambers are used, one which is open to the air and a 
    control which prevents particles from entering. Alpha particles are emmited 
    into both chambers where some of the molecules in the air are ionised 
    creating a potential difference between pairs of electrodes in the chambers. 
    The potential difference across each pair of electrodes should be the same 
    but if there is a difference, the alarm is sounded.
    \item Radioactive tracers are used to follow biochemical reactions by 
    replacing a typical chemical in the body with a radioactive isotope of the 
    same element. The fact that the radioactive decay is much more energetic
    than the chemical reactions allow the particles to be detected more easily
    using equipment like Geiger counters.
    \item Carbon dating is used to determine the age of organic material by 
    examining the carbon-14 (radiocarbon) content of the material. The half-life
    of $^{14}$C is approximately 5730 years.
\end{itemize}
\end{document}
